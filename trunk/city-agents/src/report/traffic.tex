\documentclass[a4paper,10pt]{article}
\usepackage[italian]{babel}
\usepackage[utf8]{inputenc}

\title{Traffic World}
\author{
    Carmine Dodaro, Simone Spaccarotella \\
    \texttt{\{carminedodaro, spa.simone\}@gmail.com}
}

\begin{document}

    \maketitle

    \begin{abstract}
	Traffic World è un framework multi agente, per la simulazione di sistemi
	intelligenti. La piattaforma su cui gli agenti si muoveranno è
	rappresentata da una mappa cittadina. Il nostro obiettivo è creare un sistema
	che consenta ad ogni agente di raggiungere la destinazione desiderata nel minor
	tempo possibile.
    \end{abstract}

    \section{Introduzione}
	L'ambiente è composto da una mappa cittadina, composta da strade con un senso
	di percorrenza prefissato. Ogni agente, deve compiere un determinato percorso,
	composto da un punto di partenza e un punto di arrivo nella mappa. È richiesto
	che l'agente raggiunga la sua destinazione minimizzando il percorso da fare. È
	fondamentale, inoltre, che il numero di collisioni tra agenti sia minimo.
	Quest'ultimo è il requisito con priorità più elevata. Per questa ragione, una
	forte penalità è attribuita agli agenti coinvolti in un incidente. Nel nostro
	sistema, ogni agente agisce in modo autonomo, senza l'ausilio di semafori o di
	altri sistemi intelligenti di controllo. In questo contesto, è importante
	definire una politica di scelte per ogni agente che consenta di raggiungere un
	obiettivo comune.
	
	\section{Strategie}
    In questa sezione, descriviamo il nostro approccio alla realizzazione di un
    tale sistema. Abbiamo definito un ambiente multiagente in cui ogni agente
    collabora con gli altri. In particolare, una strategia di tipo collaborativo
    è stata attuata al fine di minimizzare il numero di incidenti.
    
    \subsection{Minimizzazione Percorso}
    
    \subsection{Minimizzare Incidenti}
    La strategia di minimizzazione degli incidenti è basata su un concetto di
    collaborazione tra agenti. Infatti, l'unico punto critico, in cui possono
    avvenire collisioni tra agenti è nell'attraversamento di un incrocio. In
    particolare, durante questa fase è importante che gli agenti comunichino al
    fine di evitare incidenti. Ad ogni incrocio, gli agenti che si trovano in
    concorrenza tra loro, negoziano l'ordine di priorità di attraversamento,
    attraverso una politica basata sulle condizioni di traffico e sulla propria
    velocità.\\
    Sia $t_A$ il traffico dell'agente A e sia $t_B$ il traffico dell'agente B, A
    precede B se $t_A < t_B$.\\
    Nel caso in cui $t_A = t_B$, si prende in considerazione la velocità dei
    due agenti. Sia $s_A$ la velocità dell'agente A e sia $s_B$ la velocità
    dell'agente B, A precede B se $s_A > s_B$.\\
    Nel caso in cui le due velocità dovessero uguali, l'agente che ha iniziato
    la comunicazione, precederà gli altri.\\    
    Questo tipo di strategia ha un duplice valore. Infatti, se da una
    parte consente la minimizzazione degli incidenti, allo stesso tempo consente
    di evitare ad agenti con poco traffico nel proprio percorso, di attendere a 
    lungo per agenti con tanto traffico, ottenendo un miglioramento della misura
    di prestazione globale.
    

\end{document}